\chapter{Descrizione dello stage}

\section{Panoramica del progetto}
L'applicativo in questione è già in attività da circa un anno e ha il compito di pianificare la produzione settimanale degli ordini richiesti dalle varie aziende. Ogni azienda definisce un insieme di linee sulle quali è possibile produrre i prodotti e fornisce anche i giorni e gli orari di lavoro.
Con queste informazioni il precedente programma era in grado di fornire una pianificazione distribuita sull'arco della settimana, valutando quale fosse l'ordinamento migliore e quali ordini scartare se il tempo a disposizione non fosse stato sufficiente. Questa logica di pianificazione ometteva però il controllo della disponibilità di materie prime e\o semilavorati, la quale veniva considerata sufficiente a produrre tutti gli ordini richiesti. Non veniva inoltre considerato il tempo di produzione degli eventuali semilavorati richiesti, e non si considerava nemmeno l'arrivo di materiali per il magazzino da parte dei fornitori.
Quello che mi è stato richiesto dunque è l'integrazione delle parti sopracitate.

\section{Specifiche tecniche del problema}

Nel primo periodo di stage ho speso il tempo che avevo a disposizione studiando le componenti del problema che dovevo affrontare, ricavando un quadro generale del funzionamento logico della pianificazione della produzione. L'ostacolo principale si è rivelato essere la complessità in ambito lavorativo reale della pianificazione. Ho dovuto spendere del tempo assieme al tutor interno per riuscire ad entrare nel contesto in cui avrei dovuto lavorare, facendomi spiegare le varie sfaccettature e le scelte implementative prese. Di seguito voglio fornire un'idea del problema e di come avviene la pianificazione di un singolo prodotto finito.
Perché un prodotto venga pianificato è necessario che ci sia almeno una linea in grado di produrlo, una volta definita la linea si deve scegliere una sequenza in cui produrre l'articolo. Una sequenza definisce giorno, data di inizio e di fine e può comprendere un insieme di vincoli da soddisfare (quali lavaggi oppure ordinaria manutenzione), l'articolo verrà quindi inserito all'interno della sequenza con il relativo tempo di produzione. Ogni linea ha solitamente più sequenze nelle quali è possibile collocare l'articolo. Ogni sequenza può avere dei vincoli di dipendenza con altre sequenze.


\subsection{Implementazione}
Segue la descrizione della struttura delle principali classi che vengono impiegate nell'applicativo.

\subsubsection{Linee}
Indica l'insieme delle linee sulle quali è consentita la produzione degli ordini, ogni linea ha relativo costo orario e definisce l'insieme degli articoli che può produrre con annessa velocità di produzione.
Ogni linea è così definita:
\begin{itemize}
	\item \textbf{Codice Linea}: serve ad indicare su quale linea si vuole produrre l'articolo;
	\item \textbf{Info Articolo}: lista che indica quali articoli la linea può produrre con relativo tempo di produzione;
	\item \textbf{Info Vincolo}: lista che indica i vincoli presenti sulla linea, possono essere obbligatori o opzionali;
	
	\item \textbf{Sequenze Linea}: lista che indica quali sono le sequenze di produzione della linea;
	
	\item \textbf{Pianificazione}: lista che indica quali articoli e quali vincoli sono stati pianificati nella linea corrente;
	
	
	\item \textbf{Errori}: indica quali errori si sono presentati durante la pianificazione;
	
	\item \textbf{Calendario}: indica quali sono i giorni lavorativi con le eventuali pause.
	
\end{itemize}

\subsubsection{Sequenze}
Indica l'insieme delle sequenze di produzione presenti su ogni linea, nelle quali vengono inseriti gli ordini che sono stati pianificati.
Ogni sequenza è così definita:
\begin{itemize}
	\item \textbf{Codice Sequenza}: serve ad indicare su quale sequenza si vuole inserire l'articolo;
	\item \textbf{Giorno}: indica il giorno nel quale si vuole inserire l'articolo;
	\item \textbf{Ora inizio/Ora fine}: indica gli orari di inizio e fine della sequenza corrente;
	
	\item \textbf{Elementi}: lista che indica gli articoli e i vincoli appartenenti alla sequenza corrente.
	
\end{itemize}

\subsubsection{Ordine}
Indica come si presenta un ordine da produrre.
Ogni ordine è così definito:
\begin{itemize}
	\item \textbf{Codice Articolo}: indica il codice dell'articolo;
	\item \textbf{Tipo Articolo}: indica il tipo dell'articolo, può essere un prodotto finito, semilavorato o materia prima;
	\item \textbf{Codice Linea}: indica il livello più generale della gerarchia di classificazione di un prodotto;
	\item \textbf{Codice Settore}: indica il terzo livello di gerarchia di classificazione;
	\item \textbf{Codice Famiglia}: indica il secondo livello di gerarchia, definisce la famiglia del prodotto;
	\item \textbf{Codice SottoFamiglia}: indica il livello più caratterizzante della classificazione di un prodotto;
	
	\item \textbf{Quantità}: indica la quantità richiesta da produrre;
	
	\item \textbf{Data Spedizione}: indica la data di spedizione dell'articolo;
	
	\item \textbf{Ora spedizione}: indica l'ora di spedizione dell'articolo;
	
	\item \textbf{Data Consegna}: indica la data di consegna presso la sede del cliente;
	 
	\item \textbf{Linea preferenziale}: indica la linea preferenziale di produzione dell'articolo;
	
	\item \textbf{Anno Ordine}: indica l'anno dell'ordine in questione; 

    \item \textbf{Materie Prime}: lista che indica l'insieme delle materie prime richieste per la produzione dell'ordine;
  
   \item \textbf{Semilavorati}: lista che indica l'insieme dei semilavorati richiesti per la produzione dell'ordine;
   
   \item \textbf{Riferimento Ordine}: lista che indica l'insieme degli ordini ai quali l'ordine corrente fa riferimento per la produzione di semilavorati;
   
\end{itemize}

\subsection{Problemi logici da affrontare}
Di seguito sono presentate le problematiche che l'algoritmo deve affrontare per ottenere una corretta pianificazione della produzione.

\subsubsection{Temporali}
Qui vengono descritti tutti i problemi riguardanti i tempi di produzione.
\begin{itemize}
	\item \textbf{Data Spedizione}: devono essere rispettate le date di spedizione e di consegna degli articoli da pianificare;
	\item \textbf{Tempo minimo alla consegna}: devono essere rispettate le date di inizio produzione nel caso di ordini con scadenza a breve termine;
	
	\item \textbf{Semilavorati}: i semilavorati di un ordine devono essere pianificati prima dell'ordine stesso;
	
	\item \textbf{Ordini Fornitori}: vanno considerate le date di arrivo delle materie prime da parte dei fornitori in modo da non scartare la produzione di ordini che sarebbero producibili;
	
	\item \textbf{Giorni Lavorativi}: devono essere rispettati i giorni lavorativi senza pianificare ordini al di fuori di essi;
	\item \textbf{Orario Lavorativo}: devono essere rispettati gli orari lavorativi senza eccedere dal monte ore impostato dall'azienda che esegue la pianificazione;
	\item \textbf{Sovrapposizione Sequenze}: ogni linea può produrre una singola sequenza per volta;	
	\item \textbf{Sovrapposizione Articoli}: ogni sequenza può contenere un solo articolo senza sovrapporne degli altri nello stesso lasso di tempo.
\end{itemize}

\subsubsection{Vincoli}
Qui vengono descritti tutti i vincoli da rispettare.
\begin{itemize}
	
	\item \textbf{Materie prime}: non è possibile produrre un ordine se non sono presenti sufficienti materie prime;
	
	\item \textbf{Vincoli Obbligatori}: devono essere pianificati tutti i vincoli obbligatori di ogni sequenza;
	\item \textbf{Vincoli Condizionati}: devono essere pianificati tutti i vincoli condizionati di ogni sequenza in base alle condizioni imposte;
\item \textbf{Vincoli Opzionali}: devono essere pianificati i vincoli opzionali solo in caso ci sia una finestra temporale sufficiente altrimenti si lascia spazio alla produzione di articoli;
	\item \textbf{Vincoli Articolo}: devono essere rispettati i vincoli di produzione di ogni articolo, quali linea preferenziale o linea con maggiore velocità di produzione.
\end{itemize}

\subsubsection{Scelte implementative}
Dopo una discussione col tutor interno abbiamo scartato l'ultimo vincolo facoltativo FA2, presente nel piano di lavoro, il quale riguardava la ripianificazione degli articoli in caso di guasti o necessità aziendali. Optando per una semplice nuova esecuzione dell'applicativo con i nuovi dati interessati. 

\section{Obiettivi}

Gli obiettivi da raggiungere durante il progetto sono elencati di seguito, fanno riferimento a quanto riportato nella versione definitiva del piano di lavoro, con qualche aggiunta o modifica in sede di sviluppo, previa discussione con il tutor aziendale.
Per ogni obiettivo è definito il grado di completamento: nullo, parziale, totale.

\begin{figure}[h]
	\includegraphics[width=9cm]{res/images/requisiti.png}
	\centering
	\caption{Requisiti}
\end{figure}


\subsection{Requisiti obbligatori}
\begin{itemize}
	\item Ottimizzazione della pianificazione tenendo conto delle giacenze
	di magazzino: raggiungimento totale;
	\item Sviluppo di nuove strategie di scelta dell’Algoritmo Greedy e
	confronto dei risultati ottenuti in relazione alla funzione obiettivo: raggiungimento totale;
	\item Integrazione della Tabu Search con nuovi criteri di aspirazione e
	arresto: raggiungimento totale;
	\item Sviluppo di nuovi meccanismi di esplorazione del vicinato nella
	Tabu Search: raggiungimento totale;
	\item Acquisizione di competenze sull’utilizzo di algoritmi di Ricerca
	Operativa e applicazione in un caso di studio reale: raggiungimento totale.

	
\end{itemize}

\subsection{Requisiti desiderabili}
\begin{itemize}
	\item Ottimizzazione della pianificazione tenendo conto degli ordini a
	fornitore presenti a sistema: raggiungimento totale;
	\item Ottimizzazione della pianificazione tenendo conto dei
	semilavorati: raggiungimento totale;
\end{itemize}

\subsection{Requisiti desiderabili}
\begin{itemize}
	\item Utilizzo di multithreading nelle fasi in cui è richiesta una maggiore
	capacità di calcolo: raggiungimento totale;
	\item Data una pianificazione inserita a sistema, eseguire una
	ripianificazione considerando vincoli dovuti a necessità aziendali
	dell’ultimo momento (anomalie, guasti, ordini urgenti, etc):raggiungimento nullo.
\end{itemize}

Come accennato nella sezione precedente il requisito FA2 "Data una pianificazione inserita a sistema, eseguire una
ripianificazione considerando vincoli dovuti a necessità aziendali
dell’ultimo momento" è stato scartato a seguito di una riunione con il tutor aziendale. Siamo quindi giunti alla conclusione che, in termini di efficacia, è preferibile una nuova esecuzione dell'algoritmo a partire da zero piuttosto che conservare gli attuali dati e procedere con una nuova rielaborazione. Questo rende anche più facile la definizione e l'aggiunta dei nuovi vincoli imposti, quali linee bloccate o sequenze non più attive, che potrebbero influenzare negativamente i precedenti dati inseriti.

