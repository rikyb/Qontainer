% !TEX encoding = UTF-8
% !TEX TS-program = pdflatex
% !TEX root = ../tesi.tex

%**************************************************************
% Sommario
%**************************************************************
\cleardoublepage
\phantomsection
\pdfbookmark{Convenzioni tipografiche}{Convenzioni tipografiche}
\begingroup
\let\clearpage\relax
\let\cleardoublepage\relax
\let\cleardoublepage\relax

\chapter*{Convenzioni tipografiche}

Vengono qui definite le regole tipografiche adottate nella stesura del testo:

\begin{itemize}
    \item gli acronimi vengono definiti nel testo al primo utilizzo;

	\item le abbreviazioni, gli acronimi e i termini che possono risultare ambigui o facenti parte del linguaggio tecnico o troppo specifico vengono definiti in modo approfondito alla fine del seguente documento in una sezione chiamata glossario;
	
	\item la prima occorrenza dei termini sopracitati viene identificata da un g a pedice come segue: glossario\glo;
	
\end{itemize}

%\vfill
%
%\selectlanguage{english}
%\pdfbookmark{Abstract}{Abstract}
%\chapter*{Abstract}
%
%\selectlanguage{italian}

\endgroup			

\vfill