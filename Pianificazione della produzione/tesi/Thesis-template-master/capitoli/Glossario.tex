\chapter{Glossario}

\begin{description}
    \item \label{Alert} \textbf{Alert:} notifica di avviso di un avvenimento anomalo \hyperref[alert]{[21]}.
    \item \label{Breakpoint} \textbf{Breakpoint:} punto di interruzione forzato durante l'esecuzione di un applicativo, risulta utile nel caso in cui si debba controllare lo stato attuale delle
     variabili coinvolte nell'esecuzione \hyperref[breakpoint]{[22]}.
    \item \label{Client/Server} \textbf{Client/Server:} i sistemi client/server sono un'evoluzione dei sistemi basati sulla condivisione semplice delle risorse. La presenza di un server permette ad un
     certo numero di client di condividerne le risorse, lasciando che sia il server a gestire gli accessi alle risorse per evitare conflitti di utilizzazione tipici dei primi 
     sistemi informatici \hyperref[client/server]{[23]}.
    \item \label{Commesso viaggiatore} \textbf{Commesso viaggiatore:} il problema del commesso viaggiatore è uno
    dei casi di studio tipici dell’informatica teorica e della teoria della complessità. Il
    nome nasce dalla sua più tipica rappresentazione: data una rete di città, connesse
    tramite delle strade, trovare il percorso di minore distanza che un commesso
    viaggiatore deve seguire per visitare tutte le città una ed una sola volta. È
    stato dimostrato che TSP è un problema NP-difficile e la versione decisionale del
    problema ("dati i pesi e un numero x, decidere se ci sia una soluzione migliore dix") è NP-completa \hyperref[slide]{[16]}.
    
    \item \label{Criteri di aspirazione} \textbf{Criteri di aspirazione:} servono per far terminare l'esecuzione dell'algoritmo Tabu Search soddisfando appunto dei criteri imposti dal programmatore, basati su una scelta
    euristica, la quale permette di velocizzare la scelta della soluzione migliore possibile \hyperref[slide]{[16]}.
    
    \item \label{Data center} \textbf{Data center:} dal punto di vista infrastrutturale il data center è il cuore pulsante del business perché ospita tutte le apparecchiature che consentono di governare i processi,
    le comunicazioni e i servizi a supporto di qualsiasi attività aziendale \hyperref[datacenter]{[24]}.
    
     \item \label{DBMS} \textbf{DBMS:} \textit{Database Management System}, è un sistema software progettato per consentire la creazione, la manipolazione e l'interrogazione efficiente di database, 
    per questo detto anche "gestore o motore del database", è ospitato su architettura hardware dedicata oppure su semplice computer \hyperref[dbms]{[25]}.
    
    \item \label{Debug} \textbf{Debug:} nell'ambito dello sviluppo software, indica l'attività che consiste nell'individuazione e correzione da parte del programmatore di uno o più errori (bug) rilevati
     nel software, direttamente in fase di programmazione oppure a seguito della fase di testing o dell'utilizzo finale del programma stesso \hyperref[debug]{[26]}.
    
     \item \label{Distinta base} \textbf{Distinta base:} insieme delle componenti che formano il prodotto finito, ognuna assume un insieme di caratteristiche proprie \hyperref[distinta-base]{[27]}.
    
     \item \label{Diversificazione} \textbf{Diversificazione:} insieme di criteri che permetto all'algoritmo Tabu Search di sondare alternative diverse rispetto a quanto fatto fino ad ora \hyperref[slide]{[16]}.
    
     \item \label{Estensioni} \textbf{Estensioni:} insieme ausiliario di software che aiuta, aumenta e semplifica il processo di codifica e sviluppo software \hyperref[estensione]{[28]}.
    
     \item \label{Euristica} \textbf{Euristica:} si definisce procedimento euristico, un metodo di approccio alla soluzione dei problemi che non segue un chiaro percorso,
     ma che si affida all'intuito e allo stato temporaneo delle circostanze, al fine di generare nuova conoscenza. In particolare, l'euristica di una teoria dovrebbe indicare
    le strade e le possibilità da approfondire nel tentativo di rendere la teoria "progressiva", in grado, cioè, di garantirsi uno sviluppo empirico tale da prevedere fatti
    nuovi non noti al momento dell'elaborazione del nocciolo della teoria \hyperref[slide]{[16]}.
    \item \label{Greedy} \textbf{Greedy:} è un paradigma algoritmico, dove l'algoritmo cerca una soluzione ottima da un punto di vista globale attraverso la
     scelta della soluzione più golosa (aggressiva o avida, a seconda della traduzione preferita del termine greedy dall'inglese) a ogni passo locale. 
     Quando applicabili, questi algoritmi consentono di trovare soluzioni ottimali per determinati problemi in un tempo polinomiale, mentre negli altri non 
     è garantita la convergenza all'ottimo globale \hyperref[slide]{[16]}.
    \item \label{IDE} \textbf{IDE:} \textit{Integrated development environment}, è un software che, in fase di programmazione, supporta i programmatori nello sviluppo del codice sorgente di un programma \hyperref[ide]{[29]}.
    \item \label{Intensificazione} \textbf{Intensificazione:} insieme di criteri che permetto all'algoritmo Tabu Search di effettuare un insieme di mosse specifiche, scartandone delle
    altre, per cercare di ottenere la miglior ottimizzazione della soluzione.
    \item \label{Log} \textbf{Log:} File su cui sono registrati eventi caratteristici dell’applicazione e che fungono
    in certi casi da vero e proprio protocollo di entrata e di uscita. Ad esempio in
    programmazione il file di log evidenzia il tipo di errore e il punto nel codice
    da parte dell’IDE grazie al messaggio inviato dal canale di standard error (es. eccezioni) \hyperref[log]{[30]}.
    \item \label{Multithreading} \textbf{Multithreading:} Il multithreading migliora le prestazioni dei programmi solamente quando questi sono stati sviluppati suddividendo il carico di lavoro su più thread
     che possono essere eseguiti in apparenza in parallelo. Mentre i sistemi multiprocessore sono dotati di più unità di calcolo indipendenti per le quali l'esecuzione è effettivamente parallela, un sistema
     multithread invece è dotato di una singola unità di calcolo che si cerca di utilizzare al meglio eseguendo più thread nella stessa unità di calcolo. Le due tecniche sono complementari: a volte i sistemi
     multiprocessore implementano anche il multithreading per migliorare le prestazioni complessive del sistema \hyperref[multithread]{[31]}.
    \item \label{Np-hard} \textbf{NP-hard:} file su cui sono registrati eventi caratteristici dell’applicazione e che fungono
    in certi casi da vero e proprio protocollo di entrata e di uscita. Ad esempio in
    programmazione il file di log evidenzia il tipo di errore e il punto nel codice
    da parte dell’IDE grazie al messaggio inviato dal canale di standard error (es.
    eccezioni) \hyperref[slide]{[16]}.
    \item \label{Planning} \textbf{Planning:} processo di creazione di un piano da seguire per portare a termine gli obiettivi desiderati. È necessario fare uso di questo processo prima 
    dell'inizio della fase lavorativa \hyperref[planning]{[32]}.
    \item \label{Ricorsione} \textbf{Ricorsione:} tecnica che risulta particolarmente utile per eseguire dei compiti ripetitivi su di un set di variabili in input.
    L'algoritmo richiama se stesso generando una sequenza di chiamate che ha termine al verificarsi di una condizione particolare che viene chiamata condizione di terminazione, che in genere si ha con particolari
    valori di input \hyperref[ricorsione]{[33]}.
    \item \label{Scheduling} \textbf{Scheduling:} distribuzione temporale di un insieme di attività e processi, necessario ad indicarne inizio, durata e tempo di fine \hyperref[scheduling2]{[34]}. 
    \item \label{Tabu List} \textbf{Tabu List:} lista utilizzata per memorizzare i risultati delle soluzioni ottenute dall'algoritmo di Tabu Search o per salvare il loro processo di raggiungimento \hyperref[slide]{[16]}.
    \item \label{Tabu Search} \textbf{Tabu Search:} è una tecnica meta-euristica utilizzata per la soluzione di numerosi problemi di ottimizzazione, tra cui problemi di scheduling e routing, 
    problemi su grafi e programmazione intera \hyperref[slide]{[16]}.
    \item \label{Training} \textbf{Training:} periodo in cui il soggetto si allena all'utilizzo di nuove tecnologie e strumenti sui quali farà affidamento per i futuri lavori.
    \item \label{UI} \textbf{UI:} \textit{User interface}, è come si presenta l'applicativo all'utente finale a livello di visualizzazione \hyperref[UI]{[35]}. 
    
\end{description}
