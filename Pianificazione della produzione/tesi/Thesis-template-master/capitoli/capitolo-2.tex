% !TEX encoding = UTF-8
% !TEX TS-program = pdflatex
% !TEX root = ../tesi.tex

%**************************************************************
\chapter{Descrizione dello stage}
\label{cap:Descrizione-dello-stage}

\section{Panoramica del progetto}
L'applicativo per la pianificazione della produzione è già in attività da circa un anno ed è già sfruttato dai vari clienti di Ergon Informatica. Ogni azienda definisce un insieme di linee sulle quali è possibile produrre i prodotti e fornisce anche i giorni e gli orari di lavoro.
\\Con queste informazioni, il precedente programma era in grado di fornire una pianificazione distribuita sull'arco della settimana, valutando quale fosse l'ordinamento migliore e quali ordini scartare se il tempo a disposizione non fosse stato sufficiente. Questa logica di pianificazione ometteva però il controllo della disponibilità di materie prime e/o semilavorati,
 la quale veniva considerata sufficiente a produrre tutti gli ordini richiesti.\\ Non veniva, inoltre, considerato il tempo di produzione degli eventuali semilavorati richiesti, e non si considerava nemmeno l'arrivo di materiali per il magazzino da parte dei fornitori.
Il progetto di stage richiedeva, dunque, l'integrazione delle parti sopracitate.

\newpage
\section{Obiettivi}

Gli obiettivi da raggiungere durante il progetto sono elencati di seguito e fanno riferimento a quanto riportato nella versione definitiva del piano di lavoro, 
vengono riportati nella Tabella 2.1.
\\Ogni obiettivo è composto dalla seguente struttura:
\begin{itemize}
	\item \textbf{codice identificativo}: ogni codice identificativo è univoco e conforme alla seguente codifica:\\
	\centerline{\textbf{[Importanza][Codice]}} \\ \\
	Il significato delle cui voci è:
	\begin{itemize}
		\item \textbf{Importanza}: ogni obiettivo può assumere uno dei seguenti valori:
		\begin{itemize}
			\item \textit{OB}: obiettivo obbligatorio: irrinunciabili per qualcuno degli stakeholder;
			\item \textit{DE}: obiettivo desiderabile: non strettamente necessari ma  a valore aggiunto riconoscibile;
			\item \textit{FA}: obiettivo facoltativo: relativamente utili oppure contrattabili più avanti nel progetto.	
		\end{itemize}
		\item \textbf{Codice}: è un identificatore univoco del requisito segue un ordine incrementale.
	\end{itemize}
\end{itemize}

\renewcommand{\arraystretch}{1.5}
\rowcolors{2}{dispari}{pari}
\arrayrulecolor{white}
\begin{longtable}{ 
		>{\centering}p{0.17\textwidth} 
		>{\raggedright}p{0.60\textwidth}
		>{\raggedright}p{0.29\textwidth} 
		>{\centering}p{0.15\textwidth}
	}
	
	\caption{Tabella degli obiettivi del piano di lavoro}\\
	\rowcolorhead
	\colorhead\textbf{Codice} & \centering\colorhead\textbf{Descrizione}
	\tabularnewline
	\endfirsthead
	\rowcolor{white}\caption[]{(continua)}\\
	\rowcolorhead
	\colorhead\textbf{Codice} & \centering\colorhead\textbf{Descrizione}
	
	\tabularnewline
	\endhead
	
	%RO1---------------------------------------------------------
	\rowcolordark \textbf{OB1} & Ottimizzazione della pianificazione tenendo conto delle giacenze di magazzino 
	\tabularnewline
	\rowcolorlight \textbf{OB2} & Sviluppo di nuove strategie di scelta dell’Algoritmo \hyperref[Greedy]{Greedy\glo} e confronto dei risultati ottenuti in relazione alla funzione obiettivo
	\tabularnewline
	\rowcolordark \textbf{OB3} & Integrazione della \hyperref[Tabu Search]{Tabu Search\glo} con nuovi criteri di aspirazione e arresto
	\tabularnewline
	\rowcolorlight \textbf{OB4} & Sviluppo di nuovi meccanismi di esplorazione del vicinato nella Tabu Search
	\tabularnewline
	\rowcolordark \textbf{OB5} & Acquisizione di competenze sull’utilizzo di algoritmi di Ricerca Operativa e applicazione in un caso di studio reale
	\tabularnewline
	\rowcolorlight \textbf{DE1} & Ottimizzazione della pianificazione tenendo conto degli ordini a fornitore presenti a sistema
	\tabularnewline
	\rowcolordark \textbf{DE2} & Ottimizzazione della pianificazione tenendo conto dei semilavorati
	\tabularnewline
	\rowcolorlight \textbf{FA1} & Utilizzo di \hyperref[Multithreading]{multithreading\glo} nelle fasi in cui è richiesta una maggiore capacità di calcolo
	\tabularnewline
	\rowcolordark \textbf{FA2} & Data una pianificazione inserita a sistema, eseguire una ripianificazione considerando vincoli dovuti a necessità aziendali dell’ultimo momento (anomalie, guasti, ordini urgenti, etc.)
	\tabularnewline
	
\end{longtable}


Il requisito FA2 "Data una pianificazione inserita a sistema, eseguire una
ripianificazione considerando vincoli dovuti a necessità aziendali
dell’ultimo momento" è stato scartato a seguito di una riunione con il tutor aziendale. \\Siamo quindi giunti alla conclusione che, in termini di efficacia, è preferibile una nuova esecuzione dell'algoritmo a partire da zero piuttosto che conservare gli attuali dati e procedere con una nuova rielaborazione. Questo rende anche più facile la definizione e l'aggiunta dei nuovi vincoli imposti, quali linee bloccate o sequenze non più attive, che potrebbero influenzare negativamente i precedenti dati inseriti.


\section{Pianificazione temporale}

La seguente sezione vuole descrivere come sono state suddivise le 320 ore previste per il periodo di stage, affiancando ad ogni settimana lavorativa i requisiti e gli
 obiettivi raggiunti.\\ Il periodo di stage inizia in data 09-settembre-2019 con termine ufficiale in data 01-novembre-2019.
\\ Vengono di seguito elencate le attività previste per ogni settimana lavorativa, il totale delle ore previste per ogni attività ed eventuali
 variazioni vengono specificate nella tabella \hyperref[effettive0]{Tabella 2.2}.


 \renewcommand{\arraystretch}{1.5}
 \rowcolors{2}{dispari}{pari}
 \arrayrulecolor{white}
 \begin{longtable}{ >{}p{0.60\textwidth} >{\centering}p{0.16\textwidth}
	 >{\centering}p{0.15\textwidth} >{\centering}p{0.14\textwidth}}
	 \caption{Ore preventivate}
	 \label{effettive0}
 \\
 \rowcolorhead 
 
 \centering \textbf{\color{white}Descrizione dell'attività}
 &\textbf{\color{white}Ore preventivate nel piano di lavoro} 
 &\textbf{\color{white}Ore preventivate all'inizio dello stage} 
  
 
 \endhead	
 
 Analisi del modulo software esistente, funzionalità da realizzare e documentazione disponibile dell’algoritmo esistente	& 40 & 40	\tabularnewline
 Analisi dei requisiti e stesura della relativa documentazione	& 12 & 4	\tabularnewline
 	Studio delle tecnologie aziendali necessarie allo sviluppo del modulo & 16 & 20 	\tabularnewline
 	Studio di algoritmi e tecniche di Ricerca Operativa e Ottimizzazione Combinatoria	& 40 & 40 	\tabularnewline
  Sviluppo procedura di gestione degli ordini fornitori & 30 & \centering 30 	\tabularnewline
  Sviluppo procedura di gestione delle giacenze di magazzino & 15  & \centering 15 	\tabularnewline
  Sviluppo procedura di gestione dei semilavorati pianificati & 32  & \centering 32 	\tabularnewline
  Integrazione dei vincoli del problema di ottimizzazione con i nuovi parametri & 15  & \centering 15 	\tabularnewline
  Algoritmo Greedy: sviluppo di nuove strategie di scelta del passo successivo adottato dall’algoritmo nella costruzione della soluzione del problema & 20  & \centering 20 	\tabularnewline
  Integrazione della Tabu Search con nuovi meccanismi: criteri di aspirazione e arresto, variazione dei meccanismi di esplorazione del vicinato e adozione di tecniche di intensificazione e diversificazione & 20 & \centering 20 	\tabularnewline
  
 	Validazione e test & 60 & 60	 	\tabularnewline
	 Stesura della documentazione del prodotto sviluppato & 24 & 24	 	\tabularnewline
	 \textbf{Totale ore} & \textbf{320} & \textbf{320}
 \end{longtable}

Ogni settimana lavorativa prevede le seguenti attività:
\begin{itemize}
	\item \textbf{Prima settimana}:
	\begin{itemize}
		 \item analisi del modulo software esistente;
		 \item funzionalità da realizzare;
	 	 \item studio della documentazione disponibile dell’algoritmo esistente.
	\end{itemize}
	\item \textbf{Seconda settimana}: 
	\begin{itemize}
		\item analisi dei rischi;
		\item stesura dell'analisi dei requisiti che comprende le nuove funzionalità da integrare nel software esistente;
		\item  inizio della stesura di documentazione a supporto dell'architettura utilizzata.
	\end{itemize}
	\newpage
	\item \textbf{Terza settimana}:
		\begin{itemize}
		\item Studio delle tecnologie aziendali necessarie allo sviluppo del
		modulo in particolare linguaggio di programmazione VB.NET, componenti
		DevExpress e database Informix;
		\item  \hyperref[Training]{training\glo} sulle nuove tecnologie da utilizzare con realizzazione di un software di prova;
		\item studio di algoritmi e tecniche di Ricerca Operativa e	Ottimizzazione Combinatoria;
		\item riunioni col tutor aziendale per decidere le scelte implementative dell'algoritmo Greedy.
	\end{itemize}
	\item \textbf{Quarta settimana}:
		\begin{itemize}
		\item Algoritmo Greedy: sviluppo di nuove strategie di scelta
		del passo successivo adottato dall’algoritmo nella
		costruzione della soluzione del problema;
		\item sviluppo procedura di gestione delle giacenze di
		magazzino.
		\end{itemize}
	\item \textbf{Quinta settimana}:
		\begin{itemize}
		\item sviluppo procedura di gestione dei semilavorati
		pianificati;
			\item sviluppo procedura di gestione degli ordini fornitori.
		\end{itemize}
	\item \textbf{Sesta settimana}:
		\begin{itemize}
		\item integrazione della Tabu Search con nuovi meccanismi:
		criteri di aspirazione e arresto, variazione dei
		meccanismi di esplorazione del vicinato e adozione di
		tecniche di intensificazione e diversificazione.
		\end{itemize}
	\item \textbf{Settima settimana}:
		\begin{itemize}
		\item fase di test dei dati con valori reali forniti dai clienti di Ergon;
		\item comparazione dell'algoritmo ultimato con il precedente algoritmo in uso;
		\item verifica della bontà della soluzione in rapporto coi dati forniti dai clienti.
		\end{itemize}
	\item \textbf{Ottava settimana}:
		\begin{itemize}
		\item stesura della documentazione a supporto del prodotto sviluppato, con risalto sulle scelte implementative non banali.
		\end{itemize}
\end{itemize}

\pagebreak
\section{Analisi dei rischi}
È stata eseguita un'iniziale analisi dei rischi, i quali possono portare a problematiche in grado di far procedere a rilento la realizzazione del progetto di stage.\\
La \hyperref[rischi]{Tabella 2.3} contiene i rischi preventivati durante la seconda settimana di stage. Ogni rischio possiede un codice identificativo,
una breve descrizione affiancata dal suo rilevamento e relativo grado di rischio. Per ogni rischio è definito inoltre un piano di contingenza da seguire 
in caso di occorrenza del rischio.\\

Ogni rischio è composto dalla seguente struttura:
\begin{itemize}
	\item \textbf{codice identificativo}: ogni codice identificativo è univoco e conforme alla seguente codifica:\\
	\centerline{\textbf{R[Tipologia][Codice]}} \\ \\
	Il significato delle cui voci è:
	\begin{itemize}
		\item \textbf{Tipologia}: ogni rischio può assumere uno dei seguenti valori:
		\begin{itemize}
			\item \textit{O}: rischio organizzativo: causati dall'incapacità di saper gestire il tempo a disposizione;
			\item \textit{T}: rischio tecnologico: causati dall'incapacità dello stagista di approcciarsi ad una o più nuove tecnologie.
		\end{itemize}
		\item \textbf{Codice}: è un identificatore univoco del requisito segue un ordine incrementale.
	\end{itemize}
\end{itemize}

\renewcommand{\arraystretch}{1.5}
\rowcolors{2}{dispari}{pari}
\arrayrulecolor{white}
\begin{longtable}{ 
		>{\centering}p{0.17\textwidth} 
		>{\raggedright}p{0.28\textwidth}
		>{\raggedright}p{0.29\textwidth} 
		>{\centering}p{0.15\textwidth}
	}
	
	\caption{Tabella dei rischi di progetto}
	\label{rischi}
	\\
	\rowcolorhead
	\colorhead\textbf{Codice \\ Nome} & \centering\colorhead\textbf{Descrizione} & 
	\centering\colorhead\textbf{Piano di contingenza} & 
	\colorhead\textbf{Grado di rischio} 
	\tabularnewline
	\endfirsthead
	\rowcolor{white}\caption[]{(continua)}\\
	\rowcolorhead
	\colorhead\textbf{Nome \\ Codice} & \centering\colorhead\textbf{Descrizione} & 
	\centering\colorhead\textbf{Rilevamento} & 
	\colorhead\textbf{Grado di rischio} 
	\tabularnewline
	\endhead
	
	%RO1---------------------------------------------------------
	\rowcolordark \textbf{RO1} \\ Problematiche di Ricerca Operativa & 
	Dovendo svolgere uno studio individuale delle tematiche di Ricerca Operativa da affrontare, è possibile imbattersi in argomenti poco intuitivi o complicati. &
	Vengono fornite le dispense riguardanti il contesto in cui si andrà a lavorare e in caso di incomprensioni si farà affidamento sia sul tutor interno sia al proprio relatore. &
	Occorrenza: \textbf{Media} \\
	Pericolosità: \textbf{Alta} 
	\tabularnewline
	
	%RT1---------------------------------------------------------
	\rowcolorlight \textbf{RT1} \\ Inesperienza tecnologica & 
	Alcune delle tecnologie adottate sono nuove, pertanto è possibile incorrere in problemi durante lo svolgimento delle attività che le coinvolgono. &
	Viene fornita la documentazione ufficiale in modo da avere ampio supporto per qualsiasi lacuna di natura tecnica. Se ciò non dovesse bastare si farà affidamento al tutor interno o qualche membro delegato del team di sviluppo di Ergon. &
	Occorrenza: \textbf{Media} \\
	Pericolosità: \textbf{Media} 
	\tabularnewline

	%RT2---------------------------------------------------------
	\rowcolordark \textbf{RT2} \\ Scelte implementative & 
	Non è detto che tutte le scelte prese in considerazione durante lo studio del problema portino ad una corretta soluzione o ad una esecuzione in tempi accettabili. &
	Verranno valutate, in caso di necessità, nuove strade da percorrere per la risoluzione del problema. &
	Occorrenza: \textbf{Bassa} \\
	Pericolosità: \textbf{Alta} 
	\tabularnewline
	
	
\end{longtable}
	
\renewcommand{\arraystretch}{1}